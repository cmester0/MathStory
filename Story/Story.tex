\documentclass[11pt]{report}
\usepackage[utf8]{inputenc}
\usepackage{amsmath}
\usepackage{amssymb}
\usepackage[english]{babel}
\usepackage{fullpage}

\usepackage{graphicx}
\usepackage{subcaption}

\usepackage{xcolor} % http://ctan.org/pkg/xcolor
\usepackage{rotating}

\usepackage[hidelinks]{hyperref}

\usepackage{tikz}
\usepackage{tikz-cd}

\usepackage{todonotes}

\usepackage{forest}

\usepackage[linesnumbered,lined,boxed,commentsnumbered]{algorithm2e}

\author{Lasse Letager Hansen (201508114)}
\date{\today}
\title{Math Story}

\usepackage{proof}
\usepackage{amsthm}

\theoremstyle{plain} % default
\newtheorem{thm}{Theorem}[section]
\newtheorem{lem}[thm]{Lemma}
\newtheorem{prop}[thm]{Proposition}
\newtheorem{cor}{Corollary}

\theoremstyle{definition}
\newtheorem{axiom}[thm]{Axiom}% [section]
\newtheorem{defn}[thm]{Definition}% [section]
\newtheorem{st}[thm]{Story}% [section] 
\newtheorem{conj}{Conjecture}[section]
\newtheorem{exmp}{Example} % [section]

\theoremstyle{remark}
\newtheorem*{rem}{Remark}
\newtheorem*{note}{Note}
\newtheorem{case}{Case}

% Custom commands

\newcommand*{\thmref}[1]{\hyperref[thm:#1]{Theorem~\ref*{thm:#1}}} % Theorem~\ref
\newcommand*{\lemref}[1]{\hyperref[lem:#1]{Lemma~\ref*{lem:#1}}}
\newcommand*{\corref}[1]{\hyperref[cor:#1]{Corollary~\ref*{cor:#1}}}
\newcommand*{\defref}[1]{\hyperref[defn:#1]{Definition~\ref*{defn:#1}}}
\newcommand*{\figref}[1]{\hyperref[fig:#1]{Figure~\ref*{fig:#1}}}
\newcommand*{\tableref}[1]{\hyperref[table:#1]{Table~\ref*{table:#1}}}
\newcommand*{\exmpref}[1]{\hyperref[exmp:#1]{Example~\ref*{exmp:#1}}}
\newcommand*{\sectionref}[1]{\hyperref[sec:#1]{Section~\ref*{sec:#1}}}
\newcommand*{\subsectionref}[1]{\hyperref[sec:#1]{Subsection~\ref*{sec:#1}}}
\newcommand*{\chapterref}[1]{\hyperref[ch:#1]{Chapter~\ref*{ch:#1}}}


\newcommand{\set}[1]{\{#1\}}

\begin{document}

\maketitle

\section{TODO}
\paragraph{Theorem List}
\begin{itemize}
\item Pythagoras Theorem.
\item Midpoint Theorem.
\item Remainder Theorem.
\item Fundamental Theorem of Arithmetic.
\item Angle Bisector Theorem.
\item Inscribed Angle Theorem.
\item Ceva's Theorem.
\item Bayes' Theorem.
\end{itemize}

\chapter{Foundation}
\section{The Axioms of Mathematics}
\subsection{Set Theory}
The standard foundation of modern mathematics is set theory, or more precisely Zermelo-Fraenkel plus the Axiom of Choice (ZFC) set theory. [cite \url{https://en.wikipedia.org/wiki/Zermelo%E2%80%93Fraenkel_set_theory}]
\begin{axiom}[Axiom of extensionality]
  Two sets are equal if they have the same elements.
  \[\forall x \forall y [ \forall z (z \in x \iff z \in y) \Rightarrow x = y ] \]
\end{axiom}
\begin{st}
  % If two sets of dice have the same die, then they are equal.
\end{st}

\begin{axiom}[Axiom of regularity]
  Every non-emtpy set \(x\) contains a member \(y\) such that \(x\) and \(y\) are disjoint sets.
  \[\forall x [ \exists a ( a \in x ) \Rightarrow \exists y (y \in x \land \neg \exists z (z \in y \land z \in x)) ] \]
\end{axiom}
\begin{st}
  
\end{st}

\begin{axiom}[Axiom schema of specification]
  In general, the subset of a set \(z\) obeying a formula \(\phi(x)\) with one free variable \(x\) may be written as \(\set{x \in z : \phi(x)}\),
  \[\forall z \forall w_1 \forall w_2 \dots \forall w_n \exists y \forall x [x \in y \iff ((x \in z) \land \phi)]\]
\end{axiom}
\begin{st}
  
\end{st}

\begin{axiom}[Axiom of pairing]
  If \(x\) and \(y\) are sets, then there exists a set which contains \(x\) and \(y\) as elements
  \[\forall x \forall y \exists z ((x \in z) \land (y \in z))\]
\end{axiom}
\begin{st}
  
\end{st}

\begin{axiom}[Axiom of union]
  The axiom of union states that for any set of sets \(\mathcal{F}\) there is a set \(A\) containing every element that is a member of some member of \(\mathcal{F}\).
  \[\forall \mathcal{F} \exists A \forall Y \forall x [(x \in Y \land Y \in \mathcal{F}) \Rightarrow x \in A]\]
\end{axiom}
\begin{st}
  
\end{st}

\begin{axiom}[Axiom schema of replacement]
  Let \(\phi\) be any formula in the language of \(ZFC\) whose free variables are among \(x,y,A,w_1,\dots,w_n\), so that in particular \(B\) is not free in \(\phi\). Then:
  \[\forall A \forall w_1 \forall w_2 \dots \forall w_n [ \forall x (x \in A \Rightarrow \exists! y \phi) \Rightarrow \exists B \forall x (x \in A \Rightarrow \exists y (y \in B \land \phi)) ] \]
\end{axiom}
\begin{st}
  
\end{st}

\begin{axiom}[Axiom of infinity]
  \[\exists X [\emptyset \in X \land \forall y (y \in X \Rightarrow S(y) \in X)]\]
\end{axiom}
\begin{st}
  
\end{st}

\begin{axiom}[Axiom of power set]
  For any set \(x\)  , there is a set \(y\) that contains every subset of \(x\):
  \[\forall x \exists y \forall x [ z \subseteq x \Rightarrow z \in y ] \]
\end{axiom}
\begin{st}
  
\end{st}

\begin{axiom}[Well-ordering theorem]
  For any set \(X\), there is  a binary relation \(R\) which well-orders \(X\). This means \(R\) is a linear order on \(X\) such that every nonempty subset of \(X\) has a member which is minimal under \(R\)
  \[\forall X \exists R (R~\mathtt{well-orders}~X)\]
\end{axiom}
\begin{st}
  
\end{st}

\subsection{Type Theory}



\subsection{Category Theory}
Formalization inside topos, build on preshealfs \(\mathbf{Sets}^{C^{op}}\)

\section{Numbers}
\begin{thm}
  
\end{thm}

\paragraph{Proof}
\paragraph{Story}




\end{document}